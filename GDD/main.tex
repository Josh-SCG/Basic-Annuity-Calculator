\documentclass[a4paper, 11pt]{article}

\usepackage[english]{babel}
\usepackage[utf8]{inputenc}
\usepackage{amsmath}
\usepackage{graphicx}
\usepackage[colorinlistoftodos]{todonotes}
\usepackage{amssymb}
\usepackage{relsize}
\usepackage{MnSymbol}
\usepackage{geometry}
 \geometry{
 a4paper,
 total={170mm,257mm},
 left=20mm,
 top=20mm,
 }


\title{\huge \textbf{\textit{Living Annuity Modelling Tool}}}

\author{J-Dog}

\date{\today}


\begin{document}
\maketitle
\section*{A Game Plan for Something Not a Game:}
\subsection*{Target Audience and Concessions}
Typically those closer to retirement but as everyone will need some type of pension fund should be designed with a wide age range. With that in mind it should be usable by those that are older as well so have a simple interface and be easy to read/use.

\subsection*{General UX}
Since some might have an issue seeing how this affects them from just a ``here`s your monthly figure and total fund'' number, some form of data visualisation may go a \textit{loong} way.

\subsection*{Info on a Annuity}
This is a post-retirement product – typically one buys them with one big lump sum and draws down a monthly amount until either the money runs out, or the person dies. Whatever is left on death can be given to beneficiaries (i.e., children, spouse, whatever).
\\\\
Most places charge a larger fee when the fund starts paying out and then a yearly fee there after. Most look to be lower than 2\% though.

\section*{Assumptions and other Info}
\begin{itemize}
\item Final calculations will assume no withdrawals from new savings pot before retirement.
\item A compound interest rate of 6\%
\item Interest calculated yearly
\item An initial fee of 1.5 * VAT\% deduction (Hollard's max fee * VAT)
\item A yearly fee deduction of 1 * VAT\% (Hollard's max continual fee * VAT)
\end{itemize}

\subsubsection*{Sources}
\begin{enumerate}
\item https://personal.nedbank.co.za/learn/blog/pension-fund-vs-retirement-annuities.html
\item https://www.oldmutual.co.za/two-pot-retirement-system/
\item https://www.hollard.co.za/invest-and-save/retirement-savings/living-annuity
\end{enumerate}

\newpage
\section*{Building Plans}
A tool to model the total funds invested into a living annuity will be developed in Python using Tkinter for the interface and Matplotlib for a visualisation of the fund amount over the years.

\subsection*{Likely Input Fields}
\begin{itemize}
\item Current and proposed retirement age
\item Amount saved towards the fund
\item Current saving for the fund
\item annual funds taken out as \% or ZAR
\item Option for the 1/3 lump sum to be removed
\end{itemize}

\subsection*{Outputs}
\begin{itemize}
\item Savings at retirement
\item Monthly income
\item Years the fund can pay out
\item Visualisation
\end{itemize}

\subsection*{To Consider}
\begin{itemize}
\item Tooltips of some kind
\item ...
\end{itemize}

\end{document}